%%% ====================================================================
\documentclass[fleqn,italian]{article}
\pagestyle{headings}

\title{Calcolo dei generatori di SO(3) e SU(2)} 
\author{Francesco Pedull\`a} 
\date{Versione 0.8, 2011/05/15}

\usepackage{amsmath,amsthm}
\usepackage{graphicx}
\usepackage{verbatim}
\usepackage{amssymb,gensymb,mathrsfs,tensor,babel}


%    Some definitions useful in producing this sort of documentation:
\chardef\bslash=`\\ % p. 424, TeXbook
%    Normalized (nonbold, nonitalic) tt font, to avoid font
%    substitution warning messages if tt is used inside section
%    headings and other places where odd font combinations might
%    result.
\newcommand{\ntt}{\normalfont\ttfamily}
%    command name
\newcommand{\cn}[1]{{\protect\ntt\bslash#1}}
%    LaTeX package name
\newcommand{\pkg}[1]{{\protect\ntt#1}}
%    File name
\newcommand{\fn}[1]{{\protect\ntt#1}}
%    environment name
\newcommand{\env}[1]{{\protect\ntt#1}}
\hfuzz1pc % Don't bother to report overfull boxes if overage is < 1pc

%       Theorem environments

%% \theoremstyle{plain} %% This is the default
\newtheorem{thm}{Theorem}[section]
\newtheorem{cor}[thm]{Corollary}
\newtheorem{lem}[thm]{Lemma}
\newtheorem{prop}[thm]{Proposition}
\newtheorem{ax}{Axiom}

\theoremstyle{definition}
\newtheorem{defn}{Definition}[section]

\theoremstyle{remark}
\newtheorem{rem}{Remark}[section]
\newtheorem*{notation}{Notation}

%\numberwithin{equation}{section}

\newcommand{\thmref}[1]{Theorem~\ref{#1}}
\newcommand{\secref}[1]{\S\ref{#1}}
\newcommand{\lemref}[1]{Lemma~\ref{#1}}

\newcommand{\bysame}{\mbox{\rule{3em}{.4pt}}\,}

%       Math definitions

\newcommand{\A}{\mathcal{A}}
\newcommand{\B}{\mathcal{B}}
\newcommand{\st}{\sigma}
\newcommand{\XcY}{{(X,Y)}}
\newcommand{\SX}{{S_X}}
\newcommand{\SY}{{S_Y}}
\newcommand{\SXY}{{S_{X,Y}}}
\newcommand{\SXgYy}{{S_{X|Y}(y)}}
\newcommand{\Cw}[1]{{\hat C_#1(X|Y)}}
\newcommand{\G}{{G(X|Y)}}
\newcommand{\PY}{{P_{\mathcal{Y}}}}
\newcommand{\X}{\mathcal{X}}
\newcommand{\wt}{\widetilde}
\newcommand{\wh}{\widehat}

\DeclareMathOperator{\per}{per}
\DeclareMathOperator{\cov}{cov}
\DeclareMathOperator{\non}{non}
\DeclareMathOperator{\cf}{cf}
\DeclareMathOperator{\add}{add}
\DeclareMathOperator{\Cham}{Cham}
\DeclareMathOperator{\IM}{Im}
\DeclareMathOperator{\esssup}{ess\,sup}
\DeclareMathOperator{\meas}{meas}
\DeclareMathOperator{\seg}{seg}

\newcommand{\AMS}{American Mathematical Society}
\def\latex/{{\protect\LaTeX}}
\def\latexe/{{\protect\LaTeXe}}
\def\amslatex/{{\protect\AmS-\protect\LaTeX}}
\def\tex/{{\protect\TeX}}
\def\amstex/{{\protect\AmS-\protect\TeX}}

\newcommand{\pd}[2]{\frac{\partial{#1}}{\partial{#2}}}
\newcommand{\pdt}[2]{\frac{\partial^2{#1}}{\partial{#2}^2}}
\newcommand{\ppd}[2]{\left(\frac{\partial{#1}}{\partial{#2}}\right)}
\newcommand{\ppdt}[2]{\left(\frac{\partial^2{#1}}{\partial{#2}^2}\right)}

\newcommand{\dd}{\mathrm{d}}

\theoremstyle{plain}% default
%\newtheorem{thm}{Teorema}[section]
\newtheorem{axi}[thm]{Postulato}
%\newtheorem{lem}[thm]{Lemma}
%\newtheorem{prop}[thm]{Proposizione}
%\newtheorem*{cor}{Corollario}

\theoremstyle{definition}
%\newtheorem{defn}{Definizione}[section]
\newtheorem{conj}{Congettura}[section]
\newtheorem{exmp}{Esempio}[section]

\theoremstyle{remark}
%\newtheorem*{rem}{Osservazione}
\newtheorem*{note}{Nota}

\DeclareMathOperator{\Immag}{Im}
\DeclareMathOperator{\Real}{Re}
\DeclareMathOperator{\Div}{div}
\DeclareMathOperator{\Erg}{erg}
\DeclareMathOperator{\Sec}{sec}
\DeclareMathOperator{\Cost}{cost}
\DeclareMathOperator{\Ind}{Ind}

\newsavebox{\fmbox}
\newenvironment{fmpage}[1]
   {\begin{lrbox}{\fmbox}\begin{minipage}{#1}}
   {\end{minipage}\end{lrbox}\fbox{\usebox{\fmbox}}}

%\theoremstyle{remark}
%\newtheorem{remark}[theorem]{Remark}

%%\numberwithin{section}{section}
\numberwithin{equation}{section}
\numberwithin{thm}{section}

%    \interval is used to provide better spacing after a [ that
%    is used as a closing delimiter.
\newcommand{\interval}[1]{\mathinner{#1}}

%    Notation for an expression evaluated at a particular condition. The
%    optional argument can be used to override automatic sizing of the
%    right vert bar, e.g. \eval[\biggr]{...}_{...}
\newcommand{\eval}[2][\right]{\relax
  \ifx#1\right\relax \left.\fi#2#1\rvert}

%    Enclose the argument in vert-bar delimiters:
\newcommand{\envert}[1]{\left\lvert#1\right\rvert}
\let\abs=\envert

%    Enclose the argument in double-vert-bar delimiters:
\newcommand{\enVert}[1]{\left\lVert#1\right\rVert}
\let\norm=\enVert

\begin{document}
\maketitle
\markboth{Sample paper for the {\protect\ntt\lowercase{amsmath}} package}
{Generatori di $SO(3)$ e $SU(2)$}
\renewcommand{\sectionmark}[1]{}

\section{Introduzione}

Questo testo presenta in dettaglio il calcolo dei generatori infinitesimali 
dei gruppi di Lie $SO(3)$ e $SU(2)$. I simboli utilizzati ed i numeri delle 
equazioni si riferiscono al testo di Morton Hamermesh {\em Group Theory 
and its Application to Physical Problems}, pubblicato nel 1962. 

\section{Definizioni e notazione}

La definizione di operatore infinitesimale del gruppo [eq.(8-46)] \`e:
\begin{equation}
X_k=\sum^n_{i=1}u_{ik}(x)\pd{}{x_i}
\end{equation}
avendo definito in precedenza [eq. (8-39)]:
\begin{equation}
u_{ik}=\Biggl[\pd{f_i(x_1,\dots,x_n;a_1,\dots,a_r)}{a_k}\Biggr]_{a=0}.
\end{equation}
Notiamo che questa definizione \`e quella normalmente utilizzata nei
testi di matematica. Per le applicazioni fisiche si preferisce aggiungere
un coefficiente $-i$ che rende hermitiani anche gli operatori 
antisimmetrici reali associati alle rotazioni. Nel seguito del testo
restiamo allineati alla definizione di Hamermesh. Useremo per\`o la
seguente notazione, pi\`u compatta, in cui i vettori sono rappresentati
come vettori colonna, adottando inoltre la convenzione di usare lettere
minuscole per i vettori e maiuscole per le matrici. La trasformazione di
coordinate risulta espressa dalle $n$ equazioni
\begin{equation}
f(x;a)=\begin{pmatrix} f_1\\ . \\ . \\ f_n\end{pmatrix}
\end{equation}
dove gli argomenti delle funzioni $f_i$ sono le $n$ coordinate $x_i$ e gli
$m$ parametri (angoli) $a_i$:
\begin{equation}
v=\begin{pmatrix} x_1\\ . \\ . \\ x_n\end{pmatrix} 
\end{equation}
\begin{equation}
a=\begin{pmatrix} a_1\\ . \\ . \\ a_m\end{pmatrix}.
\end{equation}
Con questa notazione la definizione dei generatori pu\`o essere riscritta
in forma compatta (riunendo le due definizioni) 
\begin{equation}
X=(\nabla_a f^T\vert_{a=0})\nabla_x
\end{equation}
dove $\nabla_a$ indica l'operatore di derivazione rispetto ai parametri $a$,
$\nabla_x$ l'operatore di derivazione rispetto alle coordinate $x$. Notiamo
esplicitamente che il prodotto di un vettore riga per un vettore colonna 
indica il relativo prodotto tensoriale, che rappresentiamo con una matrice.

\section{Calcolo dei generatori di $SO(3)$}

Nel caso dello spazio $\mathbb{R}^3$, la trasformazione ortogonale
SO(3) pu\`o essere definita tramite il prodotto di tre matrici di 
rotazione attorno ai tre assi ortogonali $x,y,z$. Se poniamo il 
vettore dei parametri $a=(\theta,\phi,\psi)^T$, le matrici sono:
\begin{equation}
R_\theta(\theta)=\begin{pmatrix} \cos(\theta) & \sin(\theta) & 0 \\
                                -\sin(\theta) & \cos(\theta) & 0 \\
                                       0      &       0      & 1 \\
                 \end{pmatrix}
\end{equation}
\begin{equation}
R_\phi(\phi)=\begin{pmatrix} \cos(\phi) & -\sin(\phi) & 0 \\
                                 0      &     1       & 0 \\
                             \sin(\phi) & \cos(\phi)  & 0 \\
                 \end{pmatrix}
\end{equation}
\begin{equation}
R_\psi(\psi)=\begin{pmatrix} 1  &      0      &      0     \\
                             0  & \cos(\psi)  & \sin(\psi) \\
                             0  & -\sin(\psi) & \cos(\psi) \\
                 \end{pmatrix}
\end{equation}
e la funzione di trasformazione di coordinate $x'=U(a)x$ vale 
\begin{equation}
U(\theta,\phi,\psi)=R_\theta(\theta)R_\phi(\phi)R_\psi(\psi)x
\end{equation}
avendo posto $v=(x,y,z)^T$. Notiamo che i segni degli elementi di
matrice sono tali da ottenere rotazioni destrorse positive. Tale scelta
\`e arbitraria ma deve essere consistente se si vogliono ottenere risultati
direttamente confrontabili con gli operatori di rotazione che si trovano in
letteratura.
Procediamo ora al calcolo delle righe della matrice
\begin{equation}
\nabla_a f^T\vert_{a=0}=\begin{pmatrix} \partial_\theta f^T \\
                             \partial_\phi f^T \\
                             \partial_\psi f^T \\
             \end{pmatrix}_{a=0}
\end{equation}
avendo indicato con $\partial_a$ la derivata parziale rispetto ad $a$. Per
effettuare tale calcolo osserviamo che, nel derivare rispetto ad un 
qualunque angolo $a_i$, le matrici associate agli altri angoli rimangono 
costanti e vengono valutate nell'origine, per cui coincidono con l'unit\`a. 
Allora:
\begin{equation}
\begin{split}
\partial_\theta f^T\vert_{a=0} &=\partial_\theta(Uv)^T\vert_{a=0} \\
          &= \partial_\theta(v^T R_\psi^T R_\phi^T R_\theta^T)\vert_{a=0} \\
          &= v^T \partial_\theta R_\theta^T\vert_{a=0} \\
          &= (x,y,z) \begin{pmatrix}  
                          -\sin(\theta) & \cos(\theta) & 0 \\
                          -\cos(\theta) & -\sin(\theta) & 1 \\
                           0 & 0 & 0 \\ \end{pmatrix}_{a=0} \\
          &= (x,y,z) \begin{pmatrix} 
                          0 & 1 & 0 \\-1 & 0 & 0 \\ 0 & 0 & 0 \\
                          \end{pmatrix} \\
          &= (-y,x,0).
\end{split}
\end{equation}
e analogamente
\begin{equation}
\begin{split}
\partial_\phi f^T\vert_{a=0} &= v^T \partial_\phi R_\phi^T\vert_{a=0} \\
          &= (x,y,z) \begin{pmatrix}  
                 \cos(\phi) & 0 & -\sin(\phi) \\
                      0     & 1 & 0 \\
                 \sin(\phi) & 0 & \cos(\phi) \\ \end{pmatrix}_{a=0} \\
          &= (x,y,z) \begin{pmatrix} 
                          0 & 0 & -1 \\ 0 & 0 & 0 \\ 1 & 0 & 0 \\
                          \end{pmatrix} \\
          &= (z,0,-x).
\end{split}
\end{equation}
\begin{equation}
\begin{split}
\partial_\psi f^T\vert_{a=0} &= v^T \partial_\psi R_\psi^T\vert_{a=0} \\
          &= (x,y,z) \begin{pmatrix}  
                   0 &      0      & 0 \\
                   0 & -\sin(\psi) & \cos(\psi) \\
                   0 & -\cos(\psi) & \sin(\psi) \\ \end{pmatrix}_{a=0} \\
          &= (x,y,z) \begin{pmatrix} 
                          0 & 0 & 0 \\ 0 & 0 & 1 \\ 0 & -1 & 0 \\
                          \end{pmatrix} \\
          &= (0,-z,y).
\end{split}
\end{equation}
Raccogliendo i vettori nella matrice $\nabla_a f^T\vert_a$ e moltiplicando
per $\nabla_x$ si ottiene che gli operatori $X$ valgono
\begin{equation}
X=\begin{pmatrix} 0 & -z & y \\ z & 0 & -x \\ -y & x & 0 \\ \end{pmatrix}
      \begin{pmatrix} \partial_x \\ \partial_y \\ \partial_z \\ \end{pmatrix}
      = \begin{pmatrix} -z\partial_y+y\partial_z \\ 
                        z\partial_x-x\partial_z \\ 
			-y\partial_x+x\partial_y \\ \end{pmatrix}.
\end{equation}
Come atteso, coincidono (a meno di una costante moltiplicativa) con gli 
operatori di momento angolare.

\section{Calcolo dei generatori di $SU(2)$}

Nel caso dello spazio $\mathbb{C}^2$, la trasformazione ortogonale
SU(2) pu\`o essere definita dal prodotto di una matrice hermitiana
e unitaria $U$ per il vettore delle coordinate da trasformare 
\begin{equation}
v=\begin{pmatrix} z_1 \\ z_2 \\ \end{pmatrix}.
\end{equation}
Cio\`e, $U$ deve soddisfare le seguenti condizioni:
\begin{equation}
U U^\dagger = I
\end{equation}
\begin{equation}
\det(U)=1.
\end{equation}
\`E facile verificare che la matrice
\begin{equation}
U=\begin{pmatrix} \alpha & -\overline{\beta} \\ \beta & \overline{\alpha}
  \end{pmatrix}
\end{equation}
le soddisfa entrambe. Infatti
\begin{equation}
\begin{split}
U U^\dagger 
   & = \begin{pmatrix} \alpha & -\overline{\beta} \\ 
                  \beta   &  \overline{\alpha} \\ \end{pmatrix}
     \begin{pmatrix} \overline{\alpha} & \overline{\beta} \\ 
               -\beta   &  \alpha \\ \end{pmatrix} \\
   & = \begin{pmatrix} \alpha\overline{\alpha}+\beta\overline{\beta} &
                \alpha\overline{\beta}-\overline{\beta}\alpha \\
                \beta\overline{\alpha}-\overline{\alpha}\beta & 
                \beta\overline{\beta}+\overline{\alpha}\alpha \\
                \end{pmatrix} \\
   & = \begin{pmatrix} 1 & 0 \\ 0 & 1 \\ \end{pmatrix}
\end{split}
\end{equation}
dove nell'ultimo passaggio abbiamo imposto la condizione che 
\begin{equation}
det(U)=|\alpha|^2+|\beta|^2=1.
\end{equation}
Dal momento che il numero di parametri reali liberi che abbiamo \`e 8
(i 4 numeri complessi in $U$) ed abbiamo 5 equazioni (4+1 scritte sopra), ci
aspettiamo che rimangano 3 parametri reali indipendenti. In analogia
con il caso reale, indichiamo con $a=(\theta,\phi,\psi)^T$ tali parametri. 
Sempre in analogia con il caso reale e ricordando che vogliamo ottenere
matrici a traccia nulla possiamo porre
\begin{equation}
\begin{cases}
\alpha & = \sin\theta e^{i\phi} \\
\beta  & = \cos\theta e^{i\psi} .
\end{cases}
\end{equation}
\`E facile verificare che $U$, con queste posizioni, rispetta tutte le 
condizioni imposte. Quindi $U$ pu\`o essere scritta
\begin{equation}
U=\begin{pmatrix} \sin\theta e^{i\phi} & -\cos\theta e^{-i\psi} \\
                  \cos\theta e^{i\psi} & \sin\theta e^{-i\phi} \\
  \end{pmatrix}
\end{equation}
e $U^T$:
\begin{equation}
U^T=\begin{pmatrix} \sin\theta e^{i\phi} & \cos\theta e^{i\psi} \\
                  \cos\theta e^{-i\psi} & \sin\theta e^{i\phi} \\
  \end{pmatrix}.
\end{equation}
Al fine di calcolare gli operatori a partire dalla definizione, abbiamo 
bisogno della matrice $\nabla_a U^T$ per ogni parametro del vettore
 $a$. Infatti, dalla definizione abbiamo:
\begin{equation}
\nabla_a f^T=\nabla_a v^T U^T=v^T\nabla_a U^T.
\end{equation}
Il calcolo diretto di tali matrici ci d\`a:
\begin{equation}
\begin{split}
\partial_\theta U^T|_{a=0} & =
        \begin{pmatrix} \cos\theta e^{-i\phi} & -\sin\theta e^{i\psi} \\
                  -\sin\theta e^{-i\psi} & \cos\theta e^{i\phi} \\
        \end{pmatrix}_{a=0} & = &
        \begin{pmatrix} 1 & 0 \\ 0 & 1 \\ \end{pmatrix} \\
\partial_\phi U^T|_{a=0} & =
        \begin{pmatrix} -i\sin\theta e^{-i\phi} & \cos\theta e^{i\psi} \\
                  \cos\theta e^{-i\psi} & i\cos\theta e^{i\phi} \\
        \end{pmatrix}_{a=0} & = &
        \begin{pmatrix} -i & 0 \\ 0 & i \\ \end{pmatrix} \\
\partial_\psi U^T|_{a=0} & =
        \begin{pmatrix} \cos\theta e^{-i\phi} & i\sin\theta e^{i\psi} \\
                  i\sin\theta e^{-i\psi} & \cos\theta e^{i\phi} \\
        \end{pmatrix}_{a=0} & = &
        \begin{pmatrix} 1 & 0 \\ 0 & 1 \\ \end{pmatrix}
\end{split}
\end{equation}
che coincidono con le matrici di Dirac a meno di uno scambio di colonne,
trasformazione che dipende solo dalla definizione di $U$.
Infine, il calcolo completo degli operatori, in analogia al caso $SO(3)$,
ci d\`a:

\end{document}
